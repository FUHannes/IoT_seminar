\documentclass[ucs,10pt]{beamer}

\usepackage[utf8x]{inputenc}    
\usepackage[english]{babel}    

\include{fu-beamer-template}  

%\usepackage{arev,t1enc} % looks nicer than the standard sans-serif font
% if you experience problems, comment out the line above and change
% the documentclass option "9pt" to "10pt"

% image to be shown on the title page (without file extension, should be pdf or png)
\titleimage{../graphics/quantumcomputingcover}

\title[Quantum-resistant signatures for IoT] % (optional, use only with long paper titles)
{Quantum-resistant digital signatures schemes for low-power IoT}

\author % (optional, use [] for lots of authors)
{H. Hattenbach}

\institute[FU Berlin] % (optional, but mostly needed)
{Freie Universität Berlin}

\date[] % (optional, should be abbreviation of conference name)
{Seminar Internet of Things, 2021}
% - Either use conference name or its abbreviation.
% - Not really informative to the audience, more for people (including
%   yourself) who are reading the slides online

\subject{Seminar Internet of Things}
% This is only inserted into the PDF information catalog. Can be left out.

% you can redefine the text shown in the footline. use a combination of
% \insertshortauthor, \insertshortinstitute, \insertshorttitle, \insertshortdate, ...
\renewcommand{\footlinetext}{\insertshortinstitute, \insertshorttitle, \insertshortdate}

% Delete this, if you do not want the table of contents to pop up at
% the beginning of each subsection:
\AtBeginSubsection[]
{
  \begin{frame}<beamer>{Outline}
    \tableofcontents[currentsection,currentsubsection]
  \end{frame}
}

\begin{document}

\begin{frame}[plain]
  \titlepage
\end{frame}

\begin{frame}{Outline}
  \tableofcontents[pausesections]
  % You might wish to add the option [pausesections]
\end{frame}


\section{Motivation}

\begin{frame}{Quantum Computing breaks ordinary Encryption}
  % - A title should summarize the slide in an understandable fashion
  %   for anyone how does not follow everything on the slide itself.
  \begin{itemize}
  \item
    Quantum Computers operate on Qubits instead of normal Bits
  \item
    Qubits are Quantum-Mechanical
    \begin{itemize}
      \item using spin of an electrons
      \item Entanglement and Superposition
    \end{itemize}
  \item
    Algorithms can leverage those mechanics
    \begin{itemize}
      \item up to exponential speed up in some cases
      \item Shors algorithm completely breaks common Encryption 
      \begin{itemize}
        \item everythink based on Number-Theory (like RSA, ECDSA, ..)
        \item (Qubits are currently rather unstable $\rightarrow$ not broken yet)
      \end{itemize}
    \end{itemize}
  \end{itemize}
\end{frame}

\begin{frame}{Quantum Resistant Asymetric Encryption / Signatures}

  \begin{itemize}
  \item There are a few proposed solutions
    
  \item mostly based on Lattice-Based hard Problems
    \begin{itemize}
    \item Frodo-Kem (Encryption)
    \item FALCON (Signature)
    \end{itemize}
   
  \item IOT also needs to be secured
  \begin{itemize}
    \item additional challenge of beeing low power/memory
    \end{itemize}
  \end{itemize}
\end{frame}


\section{Structure}
\subsection{Skeleton}

\begin{frame}{Skeleton}
  \begin{itemize}
      \item Introduction
    
      \item Internet of Things
      
      \item Quantum Resistant Security
          \begin{itemize}
            \item Quantum Computing
            \item QR Algorithms
            \begin{itemize}
              \item Encryption
              \item Signatures
            \end{itemize}
          \end{itemize}
      
      \item QR Signatures in IoT 
      \begin{itemize}
        \item Failed Signatures
        \begin{itemize}
          \item WalnutDSA
          \item qTESLA
        \end{itemize}
        \item FALCON
      \end{itemize}
      
      
      \item {Conclusion}
  \end{itemize}
  
\end{frame}

\subsection{Width-Covorage}
\begin{frame}{Creating an Overview}
  \begin{itemize}
    \item Skimming multiple Quantum Resistant (QR) algorithms \cite{QR_algs,PQClean-GH} that focus on IoT \cite{QR_comparison,Energy_comp,QR_Iot_Lattice,QR_IoT,QR_IoT_Energy} 
    \item Deeper reserach about signature Schemes \cite{QR_sigs}
    \item and having a slightly more detailed look at two failed sschemes \cite{WalnutDSA,WalnutDSA_broken,qtesla,qtesla_masked}
    
  \end{itemize}
\end{frame}

\subsection{Depth-Covorage}
\begin{frame}{Diving Deeper}
  \begin{itemize}
    \item  having a deeper look at a NIST QR finalist with the most compact implementation:

    FALCON \cite{falcon_and_dilithium,falcon_micro_impl,bearz}
   
    \item  maybe having an outlook in the end on a Hardware-Accelerated QR chip \cite{IoT_ASIC} 
  \end{itemize}

\end{frame}

\subsection{Ressources}
\begin{frame}[allowframebreaks]{Ressources}
  \bibliographystyle{plain}
  \bibliography{../lit}
\end{frame}

\subsection{Schedule}
\begin{frame}{Planned Schedule}
\begin{itemize}
  \item 7.5. : skim breadth coverage literature
  \item 15.5.: write until signatures (at least bullet point comments)
  \item 30.5.: finish breadth (at least bullet point comments)
  \item $\alpha$: skim and bullet point Depth
  \item 30.6.: finish
\end{itemize}
\end{frame}

\end{document}

