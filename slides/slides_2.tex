\documentclass[ucs,10pt]{beamer}

\usepackage[utf8x]{inputenc}    
\usepackage[english]{babel}    

\include{fu-beamer-template}  

%\usepackage{arev,t1enc} % looks nicer than the standard sans-serif font
% if you experience problems, comment out the line above and change
% the documentclass option "9pt" to "10pt"

% image to be shown on the title page (without file extension, should be pdf or png)
\titleimage{../graphics/quantumcomputingcover}

\title[Quantum-resistant signatures for IoT] % (optional, use only with long paper titles)
{Quantum-resistant digital signatures schemes for low-power IoT}

\author % (optional, use [] for lots of authors)
{H. Hattenbach}

\institute[FU Berlin] % (optional, but mostly needed)
{Freie Universität Berlin}

\date[] % (optional, should be abbreviation of conference name)
{Seminar Internet of Things, 2021}
% - Either use conference name or its abbreviation.
% - Not really informative to the audience, more for people (including
%   yourself) who are reading the slides online

\subject{Seminar Internet of Things}
% This is only inserted into the PDF information catalog. Can be left out.

% you can redefine the text shown in the footline. use a combination of
% \insertshortauthor, \insertshortinstitute, \insertshorttitle, \insertshortdate, ...
\renewcommand{\footlinetext}{\insertshortinstitute, \insertshorttitle, \insertshortdate}

% Delete this, if you do not want the table of contents to pop up at
% the beginning of each subsection:
\AtBeginSubsection[]
{
  \begin{frame}<beamer>{Outline}
    \tableofcontents[currentsection,currentsubsection]
  \end{frame}
}


\begin{document}

\begin{frame}[plain]
  \titlepage
\end{frame}

\begin{frame}{Outline}
  \tableofcontents[pausesections]
  % You might wish to add the option [pausesections]
\end{frame}

\section{Motivation}
%1
\begin{frame}{Quantum Computing breaks ordinary Cryptography}
  % - A title should summarize the slide in an understandable fashion
  %   for anyone how does not follow everything on the slide itself.
  \begin{itemize}
  \item
    sufficiently sized Quantum Computers (explained later) on the horizon
  \item
    They can break most of the cryptography in current use
    \begin{itemize}
      \item RSA
      \item ECDSA / ECDH
      \item $\rightarrow$ Signal, WhatsApp, PGP, SSH, TLS/HTTPS, \dots
    \end{itemize}
  \item 
    not everything equally effected
    \begin{itemize}
      \item schemes in standardization to replace current cryptography
      \item some are rather computationally intense
      \item that is why i have a deeper look on which are feasable for IoT
    \end{itemize}
  \end{itemize}
\end{frame}

\subsection{Quantum Computing}
%2
\begin{frame}{Shors algorithm poses threat against asymmetric cryptography}
  \begin{itemize}
  \item
    Quantum Computers operate on Qubits instead of normal Bits
  \begin{figure}[htbp]
    \centering
    \includegraphics[width=.2\textwidth]{../graphics/The-Bloch-sphere-provides-a-useful-means-of-visualizing-the-state-of-a-single-qubit-and.png}
    \caption{Model of a qubit \cite{img:bloch}}
    \label{qubit}
  \end{figure}
  \item
    Algorithms can leverage those mechanics
    \begin{itemize}
      \item up to exponential speed up in some cases
      \item Shors algorithm completely breaks common asymmetric cryptography
      \begin{itemize}
        \item can derive private key from public key
        \item for everything based on Number-Theory (like RSA, ECDSA, ..)
      \end{itemize}
      \item  Grovers algorithm poses threat against symmetric crypto and hash-functions
      \begin{itemize}
        \item only quadratic speed-up
        \item doubling length restores security (e.g. AES128 $\mapsto$ AES256)
      \end{itemize}
    \end{itemize}
  \end{itemize}
\end{frame}
%3
\begin{frame}
  \frametitle{Shors and Grovers algorithms}
    \begin{figure}[htbp]
      \centering
      \includegraphics[width=.7\textwidth]{../graphics/grover_circuit_3qubits.png}
      \caption{Grovers Algorithm \cite{img:grover}}
      \label{grover}
    \end{figure}
  
    \begin{figure}[htbp]
      \centering
      \includegraphics[width=.5\textwidth]{../graphics/Shor's_algorithm.png}
      \caption{Shors Algorithm\cite{img:shor}}
      \label{shor}
    \end{figure}

\end{frame}

\subsection{Internet of Things}
%4
\begin{frame}{Many ressource constrained devices}
  \begin{itemize}
  \item
    Internet of Things 
  \item
    Smart-devices that are actually pretty dumb
    \begin{itemize}
      \item little memory (kilobytes to megabytes)
      \item low computing power (slow clock, small cache, etc.)
      \item limited energy ressources (battery or solar operated)
    \end{itemize}
  \item
    NIST classified into 3 classes:
  \end{itemize}
  \begin{minipage}[t]{0.4\textwidth}
    \begin{table}
      \label{IoT-classes}
      \centering
      \caption{IETF IoT Classes}
      \begin{tabular}{|l | c c|}
          \hline
          Class & RAM & Flash \\
          \hline
          C0 & $<<$ 10 KiB & $<<$ 100 KiB\\
          C1 & ~ 10 KiB & ~ 100 KiB\\
          C2 & ~ 50 KiB & ~ 250 KiB\\
          \hline
      \end{tabular} 
  \end{table}
\end{minipage}
\begin{minipage}[t]{0.4\textwidth}
  \begin{figure}[htbp]
    \centering\hspace*{1.5cm}
    \includegraphics[width=3cm,angle=90,origin=l]{../graphics/ESP32-DevKitC-32D_t (1).png}
    \caption{ESP32\cite{img:esp}}
    \label{esp32}
  \end{figure}
\end{minipage}
\end{frame}

\section{Quantum Resistant Signature Schemes}
%5
\subsection{Performance Metrics}
\begin{frame}{What makes a signature scheme better than any other?}

  \begin{itemize}
  \item length of:
  \begin{itemize}
    \item signature
    \item public key
    \item private key
  \end{itemize}
    \item time and space needed to:
    \begin{itemize}
      \item generate keys (GEN)
      \item sign a message (SIGN)
      \item verify a message (VER)
    \end{itemize}
  \item security against quantum computers and traditional attackers
  \begin{table}
    \label{QR-classes}
    \centering
    \caption{QR Security classes and their traditional counterparts as classified by the NIST}
    \begin{tabular}{|l | c|}
        \hline
        Class & security comparable to \\
        \hline
        1 & AES-128 \\
        2 & SHA256 \\
        3 & AES-192 \\
        4 & SHA384 \\
        5 & AES-256 \\
        \hline
    \end{tabular} 
\end{table}
\end{itemize}
\end{frame}

\subsection{different types}
%6
\begin{frame}{Multiple types of underlying mathematica problems}

  \begin{itemize}
    \item Super-singular isogeny based
    \begin{itemize}
      \item SIKE
      \item not well studied
    \end{itemize}
    \item Multivariate polynomial based
    \begin{itemize}
      \item Rainbow
      \item not well studied
      \item involves guessing work $\rightarrow$ not suited for low power devices
    \end{itemize}
    \item Code based
    \begin{itemize}
      \item McEliece
      \item no finalist
    \end{itemize}
    \item Hash based
    \begin{itemize}
      \item SPHINCS+ 
      \item big signatures (see next slide)
      \item very well studied
    \end{itemize}
    \item Lattice based
    \begin{itemize}
      \item FALCON, Dilithium
      \item most promising
      \item most NIST finalists
      \item most efficient
      \item not as proofed as HBS
    \end{itemize}
  \end{itemize}
  
\end{frame}


\subsubsection{HBS}
%7
\begin{frame}
  \frametitle{Hash Based Signatures}

  \begin{itemize}
    \item Bases security upon Pre-Image resistance (of hash-functions) \\$\rightarrow$Well-Studied
    \item most simple form Lamport OTS: \begin{itemize}
      \item private key: $2n$ random strings (two for each bit in digest)
      \item public key: hash of these strings
      \item sign by publishing one string for every bit in digest (either first or second)
    \end{itemize}
    \item only useable one-time $\rightarrow$ publish $x$ keys for $x$ private keys
    \begin{itemize}
      \item greatly improved by use of Merkle tree (no need for $x$ keys, only one public)
      \item but increases signature size by $\log(x)$
    \end{itemize}
    
  \end{itemize}

\end{frame}

\subsubsection{LBS}
%8
\begin{frame}
  \frametitle{Lattice Based Signatures}

  \begin{itemize}
    \item Bases security upon hardness ov CVP
    \begin{itemize}
      \item find closest vector in a (High-$d$) Lattice
      \item private key: short basis (red)
      \item public key: long basis (black)
      \item sign by providing a lattice vector close to a point on which the message would be mapped
      \item hard with long basis but easy to verify
    \end{itemize}
    \item keys are giant since high $d$ requires $\mathcal{O}(d^2)$ scalars.
    \item reduce by introducing symmetries (NTRU\footnote{N-th Degree TRUncated Polynomial Ring Units })
    \item every signature leaks information about private key
    \begin{itemize}
      \item don't give closest vector, but a close enough one
      \item best to use gauss-sampling, but cryptographically hard
    \end{itemize}
    
  \end{itemize}
  \begin{figure}[]
    \centering
    \includegraphics[height=3cm]{../graphics/1200px-Lattice-reduction.png}
    \caption{a 2D-Lattice with two bases of different length \cite{img:lattice_bases}}
    \label{lattice}
  \end{figure}

\end{frame}

\section{Comparison}
\begin{frame}[allowframebreaks]
  \frametitle{different measurements, still many fluctuations since active research}

  \begin{table}%[htbp]
    \caption{Stack usage}
    \label{t:stack_comp}
    \centering\begin{tabular}{| r | c c c |}
        \hline
        Implementation name                     & GEN (bytes) & SIGN (bytes) & VER (bytes)\\
        \hline
        Dilithium-3 \cite{QR_Iot_Lattice}       & 50k       & 86k   & 54k\\
        2021 Dilithium(dyn)\cite{update_sign}   & -         & 52k   & 36k\\
        2021 Dilithium(sta)\cite{update_sign}   & / \footnote{precomputed key and directly stored in flash} & 35k   & 19k\\ %done: footnote visable?
        qTESLA-1 \cite{QR_Iot_Lattice}          & 22k       & 29k   & 23k\\
        qTESLA-3 \cite{QR_Iot_Lattice}          & 43k       & 28k   & 45k\\
        Falcon-5  \cite{QR_Iot_Lattice}         & 120k      & 120k  & 120k\\
        2021 FALCON \cite{update_sign}          & -         & 42k   & 4.7k\\
        \hline
    \end{tabular}
\end{table}

\begin{table}%[htbp]
    \caption{clock cycles}
    \label{t:clockcycles_comp}
    \centering\begin{tabular}{| r | c c c |}
        \hline
        Implementation name                     & GEN           & SIGN         & VER \\
        \hline
        Dilithium-3 \cite{QR_Iot_Lattice}       & 2.3           & 8.3          & 2.3 \\
        Dilithium-3 \cite{Energy_comp}          & 2.1           & 7.2          & 2.1 \\ %TODO unclear major differences betwwen different measurements, should have conducted my own but no time -> future work
        2021 Dilithium(dyn)\cite{update_sign}   & -             & 29           & 3.4\\
        2021 Dilithium(sta)\cite{update_sign}   & -             & 8            & 1.5\\
        qTESLA-3 \cite{QR_Iot_Lattice}          & 30            & 11           & 2.2\\
        Falcon-5 \cite{QR_Iot_Lattice}          & 365           & 165          & 1\\
        2021 Falcon  \cite{update_sign}         & -             & 75           & 1 \footnote{after optimizations these could be improved by further 43\% \cite{falcon_micro_impl}}\\
        \hline
    \end{tabular}
    
\end{table}

\begin{table}%[htbp]
    \caption{Flash sizes)}
    \label{t:flashsize_comp}
    \centering\begin{tabular}{| r | c |}
        \hline
        Scheme & Size \\
        \hline
        FALCON & 57KB \\
        2021 Dilithium (Dyn) & 11KB\\
        2021 Dilithium (Sta) & 26KB\\
        \hline
    \end{tabular}
\end{table}

\begin{table}%[]
    \caption{key and signature sizes}
    \label{t:key_sig_comp}
    \centering\begin{tabular}{ | r | c c | }
        \hline
        Scheme & public key & signature \\
        \hline
        SPHINCS     & 1KB   & 43KB \\
        Dilithium-3 & 1.4KB & 2.7KB\\
        FALCON-1    & 900B  & 690B\\
        FALCON-5    & 1.7KB & 1.3KB\\
        \hline
        ECDSA       & 64B   & 64B\\
        \hline
    \end{tabular}
\end{table}

\end{frame}

\subsection{FALCON}
\begin{frame}
  \frametitle{FALCON is the fastest verifier}

  \begin{itemize}
    \item most efficient by far for verification 
    \begin{itemize}
      \item smallest public key
      \item smallest signature
      \item fastest to verify
    \end{itemize}
    \item great for verification only actors
    \item signing takes very long (1s)
    \begin{itemize}
      \item since gauss sampling is used
      \item also vulnerable to timing / side channel attacks (shown effective)
    \end{itemize}
  \end{itemize}

\end{frame}

\subsection{Dilithium}
\begin{frame}
  \frametitle{Dilithium is the best allrounder}

  \begin{itemize}
    \item also great verification efficiency 
    \item ditched gauss sampling
    \begin{itemize}
      \item no FFT or FPA 
      \item everything in constant time $\rightarrow$ no timing attacks
    \end{itemize}
  \end{itemize}

\end{frame}

\section{Conclusion}
\begin{frame}
  \frametitle{QR IoT is possible}
  \begin{itemize}
    \item two viable contenders for QR signatures in IOT:
    \begin{itemize}
      \item Dilithium
      \item FALCON
    \end{itemize}
    \item already implemented with some kind of optimization
    \item still probably a little way up to key-length of ECDSA
    \item but already feasable for C2 devices and FALCON VER on C1
  \end{itemize}
  

\end{frame}

\section{Ressources}
\begin{frame}[allowframebreaks]{Ressources}
  \nocite{*}
  \bibliographystyle{plain}
  %\bibliography{img}
  \bibliography{../graphics/img.bib, ../lit.bib}
\end{frame}


\end{document}

