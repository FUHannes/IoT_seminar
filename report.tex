\documentclass[conference]{IEEEtran}
%\IEEEoverridecommandlockouts % The preceding line is only needed to identify funding in the first footnote. If that is unneeded, please comment it out.
\usepackage{cite}
\usepackage{amsmath,amssymb,amsfonts}
\usepackage{algorithmic}
\usepackage{graphicx}
\usepackage{textcomp}
\usepackage{xcolor}
\def\BibTeX{{\rm B\kern-.05em{\sc i\kern-.025em b}\kern-.08em
    T\kern-.1667em\lower.7ex\hbox{E}\kern-.125emX}}
\begin{document}

\title{Quantum-resistant digital signatures schemes for low-power IoT}

\author{\IEEEauthorblockN{1\textsuperscript{st} Hannes Hattenbach}
\IEEEauthorblockA{\textit{Computational Science} \\
\textit{Freie Universität}\\
Berlin, DE \\
hannes.hattenbach@fu-berlin.de}
}

\maketitle

\begin{abstract}
%*CRITICAL: Do Not Use Symbols, Special Characters, Footnotes, or Math in Paper Title or Abstract.
\end{abstract}

\begin{IEEEkeywords}
Internet of Things, Quantum Resistance, Secure Signatures, Power Constraint Devices
\end{IEEEkeywords}

%TODO 12 Seiten

\section{Introduction}

\section{Internet of Things}
%discussion what the boundries of iot compare different devices from linux devices rPI over 32bit Arm based processors with RTOS to 8bit microcontrollers like arduino/atmega

\section{Quantum Resistant Security}
\subsection{Quantum Computing}
\subsection{QR Algorithms}
\subsubsection{Encryption}
\subsubsection{Signatures} 
\subsubsection{Performance Metrics}

\section{QR Signatures in IoT}
\subsubsection{Performance Metrics in IoT}
\subsection{Failed Signatures}
\subsubsection{WalnutDSA}
\subsubsection{qTESLA}
\subsection{FALCON}

\section{Conclusion}

\bibliographystyle{IEEEtran}
\bibliography{lit.bib}

\end{document}
