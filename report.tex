\documentclass[conference]{IEEEtran}
%\documentclass[12pt,a4paper,english]{report}
%\IEEEoverridecommandlockouts % The preceding line is only needed to identify funding in the first footnote. If that is unneeded, please comment it out.
\usepackage{cite}
\usepackage{amsmath,amssymb,amsfonts}
\usepackage{algorithmic}
\usepackage{graphicx}
\usepackage{textcomp}
\usepackage{xcolor}
\newcommand{\comment}[1]{}
\def\BibTeX{{\rm B\kern-.05em{\sc i\kern-.025em b}\kern-.08em
    T\kern-.1667em\lower.7ex\hbox{E}\kern-.125emX}}
\begin{document}

\title{Quantum-resistant digital signatures schemes for low-power IoT}

\author{\IEEEauthorblockN{1\textsuperscript{st} Hannes Hattenbach}
\IEEEauthorblockA{\textit{Computational Science} \\
\textit{Freie Universität}\\
Berlin, DE \\
hannes.hattenbach@fu-berlin.de}
}

\maketitle

\begin{abstract}
%*CRITICAL: Do Not Use Symbols, Special Characters, Footnotes, or Math in Paper Title or Abstract.
\end{abstract}

\begin{IEEEkeywords}
Internet of Things, Quantum Resistance, Secure Signatures, Power Constraint Devices
\end{IEEEkeywords}

%TODO 12 Pages

\section{Introduction}
\comment{ %%%%%%%% NOTES %%%%%%%%%%%
%done
    „principles of data integrity, message authentication, and nonrepudiation, are going to have profound aftermath on sensory data in terms of security and privacy.“ \cite{QR_sigs}
} %%%%%%%%%%%%%%%% END %%%%%%%%%%%%%

The quantum revolution is coming. With quantum computers\footnote{compare section \ref{l:quantum_computing}} on the way to get more and more functional, people are fearing a loss of their security and privacy.
Or as \cite{QR_sigs} puts it, ``principles of data integrity, message authentication, and nonrepudiation, are going to have profound aftermath on sensory data in terms of security and privacy.''
That is because there are algorithms based on Shors algorithm that can forge signatures and decrypt encrypted messages whos security is based on discrete logarithms, including elliptic curves or prime factorization, like our most common schemes ECDSA and RSA respectively are.
The quantum computer only needs access to the public keys of these asymmetric schemes.
The expenditure to forge a signature\footnote{that is considered secure under normal circumstances 
%noTODO: more precise? Unforgeable under chosen blabla
} with classic\footnote{we refer to classic if something is not directly leveraging entanglement or superposition} computers rises exponentially with increased key length, therefor being essentially unbreakable by classic computers.
A sufficient quantum computer on the other hand can derive a private key from a public key in polynomial time, therefor rendering these schemes broken.

That is why there are currently schemes under standardization\cite{PQClean-GH} that are based on other hard problems (not number theory) like so called lattice problems that cannot be that easily forged by quantum computers to save our privacy and security.

One of the use cases not directly coming to mind for the end user, but being as important non the less is signing sensitive sensor data in the Internet of Things (IoT).
Another problem coming up in the IoT compared to end-user-devices like Laptops and Smartphones though is the severe resource constraint-ness. 
The IoT consist of low power devices with very few storage and computing power.

In this paper i am going to evaluate existing signature schemes and their usage possibilities for the IoT regarding their performance metrics.

Therefor i am going to give a small introduction and background to quantum computing, being a little more detailed about their ability to break current encryption and signature standards.
In the next section i will give an overview over current candidates for Quantum Resistant (QR) Algorithms and giving performance metrics for those.
The following chapter will then focus on signature schemes in the IoT, starting with additional performance metrics relevant in the IoT.
With a little more details about two failed signature schemes to highlight potential pitfalls. 
And finally focussing on the best signature contender for the IoT so far: FALCON.

\section{Background}
\subsection{Cryptography}
% done : already distinguish between : hashing, encryption (symmetric/asymmetric), signatures
Loosely speaking the main topic of cryptography can be divided into three groups.
The first of these groups is about one way functions, that shall not, as the name implies, be efficiently reversible.
If we create a smaller value of constant length from a bigger set of possibly variable length, we commonly refer to that as hashing.
Cryptographic hashing is important for a variety of different applications like storing and matching passwords without the ability to infer any knowledge about that password.
Hashing itself can be used for the next pillar of cryptography: signatures.
Signature schemes are used to proof integrity or authenticity of any data.
A signature scheme consists of two parts, signing and verifying. 
The last group is encryption, which ensures privacy/confidentiality of any data, s.t. only the right entities can decrypt this data.
These schemes consist of the two parts encryption and decryption.
Additionally to those parts for signatures as well as encryption there needs to be process of key-generation.
We also differentiate between symmetric and asymmetric schemes. The first one has a different private and public key while the latter uses the same for de- and encryption.
More details about which of those schemes will be more or less endangered by quantum computing are in section \ref{l:quantum_computing} and \ref{l:qr-algs}.

% noTODO ? Background section to put this in?
\subsection{Internet of Things}
\comment{ %%%%%%%% NOTES %%%%%%%%%%%
- growing: over 3 billion rn \cite{QR_IoT}
} %%%%%%%%%%%%%%%% END %%%%%%%%%%%%%

%discussion what the boundaries of iot compare different devices from linux devices rPI over 32bit Arm based processors with RTOS to 8bit micro controller like arduino/atmega
The IoT consists of devices of all sorts, having in common, that they communicate with each other and the environment rather than directly with humans.
Those devices range from automatic lights and smart home devices to tiny interconnected sensors in automatic fabrication.
A common characteristic though is, that most of these devices have limited processing power, flash storage and random access memory (RAM). 
A popular example for hobbyist IoT devices is the ESP32 from Espressif Microsystems.
They offer multiple Modules with up to 240Mhz Clock on the 32 IC, up to 16MiB Flash Storage and 320KiB RAM.
Which is more than other comparable devices but way less then a lower spec modern smartphone, with 10 times the frequency, 4GB of RAM and 64GB of storage.

Since the IoT consists of very different types of constrained nodes the IETF introduced different classes on which to classify IoT nodes, those can be seen in table \ref{IoT-classes}

\begin{table}
    \label{IoT-classes}
    \centering
    \caption{IETF IoT Classes}
    \begin{tabular}{l | c c}
        Class & RAM & Flash \\
        \hline
        C0 & $<<$ 10 KiB & $<<$ 100 KiB\\
        C1 & ~ 10 KiB & ~ 100 KiB\\
        C2 & ~ 50 KiB & ~ 250 KiB\\
    \end{tabular} 
\end{table}



\section{Quantum Resistant Security}
% XXX : more details about Shor and Grover
\subsection{Quantum Computing}\label{l:quantum_computing}
In contrast to classical computers, where information is processed in discrete states, a quantum computer leverages quantum mechanics to operate on so-called qubits - quantum objects that can be in superposition or entangled with each other. 
Opening a new kind of computing. 
One of the implications of that is, that it is now possible to factor large numbers in polynomial time using an algorithm developed by Shor \cite{Shor}. 
This algorithm uses a so-called Quantum-Fourier-Transform (QFT) to (probabilistically) get the frequencies of which a given function output occurs. That can be used together with euclids algorithm of finding the greatest common devisor to derive the prime factors. 
Prior to to quantum computers this was considered a hard problem that could only be computed in exponential time and was therefor considered practically impossible and was used as the basis-problem for RSA encryption.
Similar to that other common schemes like ECDSA can also be broken be slightly modified versions of Shors Algorithm.
\subsection{QR Algorithms}\label{l:qr-algs}
The two main algorithms with practical use cases that have a great speed-up compared to classical solutions, are the already introduced algorithm by Shor and an algorithm by Grover that can essentially reverses one-way functions by creating a superposition over all possible inputs, flipping all inputs with the wanted output (without knowing the inputs) and then flipping this state about its mean and repeating this process a lot of times \cite{Grover}.
While Shors algorithm provides exponential speed-up, Grovers algorithm only provides quadratic speed-up. It was also shown, that something similar to grovers algorithm but with exponential speedup is impossible \cite{Strengths&Weaknesses_QC}. 
Which implies that Hashing as well as symmetric cryptography stays relatively secure.
The quadratic speedup provided by quantum computers can easily be mitigated by doubling the key length.
On the other hand though, classical asymmetric cryptography is endangered by shors algorithm and quantum computers.

But not all asymmetric cryptography schemes are equally affected.
There are different proposals, both for QR encryption and for QR signature schemes.
They all do have in common though, that their security is not absolutely mathematically proven, but based upon assumptions.
We therefor need to consider a few measures that make schemes more or less secure.
\subsubsection{Performance Metrics}
\comment{ %%%%%%%% NOTES %%%%%%%%%%%
%done
    - Security Level (1-5, AES-128, SHA256, AES-192, SHA384, AES-256) \cite{QR_Iot_Lattice,Energy_comp}([7]) determined via grovers alg
    - no standard benchmark for quantum resistance \cite{QR_comparison} (NIST levels 1-5)
} %%%%%%%%%%%%%%%% END %%%%%%%%%%%%%

Some performance metrics exist in QR schemes as well as in classic schemes.

Key length and key exchange message length \cite{QR_algs} are the more obvious ones.
The computing time also comes to mind as a performance metric. Here you need to differentiate between key generation, which is less important, since it should only occur rarely, and signing as well as signature verification \footnote{as well as its counterparts de- and encryption}.

Primarily in signatures another metric arises: how often can a private key be used before it needs to be switched out for another one, because the signature leaked information of the key.
This is not particularly relevant in most cases, as methods can be used to create long term procedures from short term procedures (those where a key can rarely, if ever, be recycled).
But it is relevant in the case of the IoT, since those methods require extra memory which is sparse in IoT-devices. Additionally they tend to make the signatures themselves longer, which also is not preferable in the IoT. \cite{QR_algs}

Additionally to more traditional performance metrics we somehow need to measure the security of given schemes against an attack by a quantum computer.
Sadly there is currently no standard benchmark to measure quantum resistance \cite{QR_comparison}, nevertheless the NIST created a standard that describes how secure a scheme is against a quantum computer by classifying it within 5 classes that can be determined with grovers algorithm \cite{QR_Iot_Lattice,Energy_comp}.
Those classes can be seen in table \ref{QR-classes}

\begin{table}
    \label{QR-classes}
    \centering
    \caption{QR Security classes and their traditional counterparts as classified by the NIST}
    \begin{tabular}{l | c}
        Class & security compareable to \\
        \hline
        1 & AES-128 \\
        2 & SHA256 \\
        3 & AES-192 \\
        4 & SHA384 \\
        5 & AES-256 \\
    \end{tabular} 
\end{table}

\subsubsection{Encryption}
\comment{ %%%%%%%% NOTES %%%%%%%%%%%
%done:
    knapsack problem - broken
    ''
    conjugacy search problem and related problems in braid groups, and the problem of solving
    multivariate systems of polynomials in finite fields
    '' also mostly broken or badly understood \cite{QR_algs}

    lattice based:
    - NTRUEncrypt (compare sigs)
    code based:
    - McEliece Error correction codes transformed - secure and fast (100micros) but keys are k*n matrices : millions of bits \cite{QR_algs} - not feasable 
    multivariate-based: decryption inefficent (''guess work'') \cite{QR_comparison}
    - Rainbow gigantic 22kbyte pubk
    Supersingular EC:
    not much in use and not super researched , one impl (SIKE) \cite{QR_comparison}
    Mixed schemes for backwards comp: neither fully safe nor efficent since 2 schemes need to be saved on device \cite{QR_comparison}

mostly code/lattice based implementations \cite{QR_comparison}
} %%%%%%%%%%%%%%%% END %%%%%%%%%%%%%
QR encryption schemes can be based upon a multitude of different mathematical problems thought to be hard even for quantum computers.

Sadly, being thought of as secure mostly is not based upon actual rigorous proof but assumptions.
Therefor one problem that was used as a asymmetric encryption basis, the knapsack problem, was broken soon after its introduction by so-called approximate lattice reduction attacks \cite{QR_algs}.

Later iterations which include ``conjugacy search problem and related problems in braid groups, and the problem of solving
multivariate systems of polynomials in finite fields''\cite{QR_algs} have been under active research with the latter being broken after standardization and implementation \cite{QR_algs}.

Nevertheless there is an implementation of a multivariate-based scheme, called Rainbow, that is also currently a contender for standardization. But as an encryption scheme its not very suitable since the process of decrypting in multivariate based schemes requires some guessing work \cite{QR_comparison} which is essentially bad in IoT enviroments.
An additional problem that would make rainbow unsuitable for IoT use-cases is its big 22kB public key. While private keys can rather easily be shrunk in key-generation through help of a pseudo-random-generator, thats generally not the case for large public keys.

On the other hand we have a problem that is not yet very well researched and also not much in use, but has one implementation called SIKE. This problem is based upon supersingular elliptic Curves, which are itself a modification of elliptic curve problems that should make it quantum resistant. But since this topic isnt well-studied yet we are mostly left with Schmes based upon the following two thought-to-be quantum-hard problems.

The first one is so called code-based cryptography. Here the decoder has to correct errors of data that has been seemingly randomly shuffled, but only those with access to the private key can easily `unshuffle' the data to then use special error correction codes. 
The most researched one is called McEliece and even has quite fast ($100 \mu s$) and secure implementations. 
The main problem is, that the `shuffling' is realized through $k*n$ matrices that are generally big (millions of bits) and therefor unfeasible for constrained IoT devices.

The second one will be discussed in greater detail in section \ref{QR signatures}, since it is also used as one of the main problems for signature schemes.
Those schemes are called lattice based and also have some implementation with the most famous for encryption being NTRUEncrypt.

\subsubsection{Signatures} \label{QR signatures}
\comment{ %%%%%%%% NOTES %%%%%%%%%%%

Signatures classify as
- Hash based
- Lattice based
- Multivariate polynomial based
- Code based
- Super-singular sogen based
\cite{QR_sigs}

hash based or lattice based:
- hash: XMSS, SPHINCS \cite{QR_IoT} (WOTS in IOTAs Tangle)
- lamport OT signature \cite{QR_algs} (as the name implies only useable one time - useless ; can precompute a bunch that are verifiable with the same pubk - useless since rare storage)
- lattice based signatures (high-d basis find shortest vector (SVP) or closest lattice vector to arbitrary point (CVP)) require megabits of basis - unfeasable NTRU reduce to kilobits by introducing symetry \cite{QR_algs,QR_comparison}
    -10-100* faster than conventional crypto
    Encryption : - vulnerable to CCA , lattice reduction techniques - padding scheme [30] , longer keys \cite{QR_algs}
    Signatures : - map message to vector, sign by solving (CVP) - leaks information about PrivK - broken after 400messages - dont give closest vektor but a close enough one - secure for 1 mrd (billion) sigs (still adviced to swap after 10mil) (okayish iniot since not so many massages)
    - proposals (2017: GPV, GLP, BLISS \cite{QR_IoT})

HBS vs Lattice:
    - SPHINCS (HBS) (intel XEON (wtf?!))
        - sign: 50mil clock cycles 
        - Ver: 1.6mil
        - Key-size: pub 1 ; private 1
        - Sig size: 41KB
    - BLISS (lattice) (arm M4)
        - 5.9mil
        - 1mil
        - 7; 2 (hm ouf)
        - 0.96

	HBS is well studied/understood/practiced lattice not and vulnerabilities discovered one after the other

	Stateful (wOTS, XMSS, LMS ..)
		„stateful digital signature scheme necessitates the main- tenance of the updated nonrepeated secret key upon each signature generation process. It is essential to keep track of nonrepeated key pairs, failing which will result in the degra- dation of the security of the cryptographic scheme“ ..sounds kinda whack to me

		MSS stores only prng seed and uses merkle trees -> managing state (used keys) on other side (because reusing key is imperative to security)

	Stateless(Sphincs)
		more expensive sig gen , since key pairs used in random order - BDS optimization no longer applicable

	Stateful if performance time and processor constrained, Stateless if energy/memory constrained
\cite{QR_sigs}


 code-based:   
- McEliece decrypt padded message digest - try thousands of paddings - signing takes 30secs, 4mb priv/pub KEY-size -not feasbale
} %%%%%%%%%%%%%%%% END %%%%%%%%%%%%%
The other pillar of cryptography, signature schemes, is what we will focus on in greater detail.
As well as in encryption schemes we can differentiate between different underlying mathematical problems. Those are pretty much the same as in encryption schemes: 
Hash based,
Lattice based,
Multivariate polynomial based,
Code based,
Super-singular isogeny based.

Rainbow is the only implementation of a QR signature that is a current contender for standardization that is neither lattice nor hash based.
And as already mentioned in the previous section it is multivariate based.

Since this sparsity of alternatives we we also focus on hash and lattice based signatures in this paper.

Hash based signatures have their security based upon the hardness of reversing Hashes or one-way functions.
The most easy one is the Lamport one time signature (OTS).
That signature has essentially two private keys for every bit in the message digest. 
The advantage of those schemes is, that the private keys do not have to have any special characteristic that could be taken advantage of by a quantum computer to break anything.
They do have to be high entropy though, to not be easily forgeable with even a classic computer. These secrets are then hashed and published as the public key.
When a message is signed the signer just publishes the secret corresponding to every bit of the digest s.t. everyone can hash that secret and see that this private keys are indeed the ones corresponding to the public key.
A directly visable disadvantage is, that this schemes (as the name implies) can trivially be only used one time, since most of the private keys are now public.
%TODO: Stateful und einfach nächsten key mitsenden uns mitsignieren
%TODO: winterlitz
%TODO: stateful vs stateless and how two


\section{QR Signatures in IoT}
\comment{ %%%%%%%% NOTES %%%%%%%%%%%
Stack usage:
    name            & KeyGen    & Sign  & Verify
    Dilithium-3     & 50k       & 86k   & 54k
    newer dil(dyn)  & not meas  & 52k   & 36k \cite{update_sign}
    newer dil(sta)  & aheadOf t & 35k   & 19k \cite{update_sign}
    qTESLA-1        & 22k       & 29k   & 23k
    qTESLA-3        & 43k       & 28k   & 45k
    Falcon-5        & 120k      & 120k  & 120k
    newer FALCON    & not meas  & 42k   & 4.7k \cite{update_sign}

Clock cycles (10mil ~ 60ms (ARM M4 (168Mhz))):
    name        & KeyGen    & Sign  & Verify
    Dilithium-3 & 2.3m      & 8.3m  & 2.3m
    Dilithium-3 & 2.1       & 7.2   & 2.1 \cite{Energy_comp}
    Falcon-5    & 365m      & 165m  & 1m
    Falcon      & note meas & 75m   & 1m \cite{update_sign}
    qTESLA-3    & 30m       & 11m   & 2.2m
    \cite{QR_Iot_Lattice}

Hash-Based Sphincs promising since stateless, but many parameters to set \cite{QR_IoT_Energy}

as of \cite{QR_comparison} only schemes (out of ~50) with < 4kbit: SIKE and Round5 
} %%%%%%%%%%%%%%%% END %%%%%%%%%%%%%

\subsubsection{Performance Metrics in IoT}
\comment{ %%%%%%%% NOTES %%%%%%%%%%%
- key/ exchange message/ signature size 
- cache/ ram usage

- setup(ms)  lifetime, pubk size, privk size, sig

'' small sized public key, small digital 
signature and a range of supported hash output sizes is 
recommended''\cite{QR_Iot_Lattice}

- Stateful/less
- „signature and/or key sizes to running times and memory consumption to energy consumption „
- „From the software benchmark perspec- tive, the runtime of key generation, signing, and verification processes whereas from the hardware perspective, CPU cycles, key size, signature size, and energy consumption are among the targeted evaluation metrics. In general, the parameter sets are highly dependent on the underlying construction of a particular scheme.“
\cite{QR_sigs}

- most impls have quite large keys \cite{QR_comparison}
- key gen performance since many schemes have limited signatures \cite{QR_comparison}

- IoT evolves, when fast quantum is available iot will be better too  \cite{QR_comparison}
} %%%%%%%%%%%%%%%% END %%%%%%%%%%%%%


%\subsection{Failed Signatures}
\comment{ %%%%%%%% NOTES %%%%%%%%%%%
} %%%%%%%%%%%%%%%% END %%%%%%%%%%%%%

%\subsubsection{WalnutDSA}
\comment{ %%%%%%%% NOTES %%%%%%%%%%%
wahrscheinlich skippen da mir da hintergrund zu braid groups etc fehlt
} %%%%%%%%%%%%%%%% END %%%%%%%%%%%%%

\subsection{qTESLA}
\comment{ %%%%%%%% NOTES %%%%%%%%%%%
not in the endgame but also not broken afaik
} %%%%%%%%%%%%%%%% END %%%%%%%%%%%%%

\subsection{FALCON}
\comment{ %%%%%%%% NOTES %%%%%%%%%%%
falcon-512 (L1):
pubk/sig 897/690 bytes (dil3: 1472/2701 ecdsa: 64)
keygen: 182m clk , 118mJ (dil3: 2.3m / 1.7mJ ecdsa 5mJ)
sign/ver: 23.5/0.345 mJ (dil3 5mJ/1.7mJ ecdsa 4mJ)

falcon-1024 (L5):
pubk/sig 1793/1330 bytes
keygen: 380m clk , 232mJ
sign/ver: 45.5/0.69 mJ
\cite{Energy_comp}

} %%%%%%%%%%%%%%%% END %%%%%%%%%%%%%


\section{Conclusion}
\comment{ %%%%%%%% NOTES %%%%%%%%%%%
- of course no protection against side channel etc 
- quantum fast evolving, active field of research
- smart home, smart campus, smart city
- quantum key distribution
\cite{QR_comparison}
} %%%%%%%%%%%%%%%% END %%%%%%%%%%%%%

\bibliographystyle{IEEEtran}
\bibliography{lit.bib, lit_specific.bib}

\end{document}
