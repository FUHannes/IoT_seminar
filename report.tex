\documentclass[conference]{IEEEtran}
%\documentclass[12pt,a4paper,english]{report}
%\IEEEoverridecommandlockouts % The preceding line is only needed to identify funding in the first footnote. If that is unneeded, please comment it out.
\usepackage{cite}
\usepackage{amsmath,amssymb,amsfonts}
\usepackage{algorithmic}
\usepackage{graphicx}
\usepackage{textcomp}
\usepackage{xcolor}
\def\BibTeX{{\rm B\kern-.05em{\sc i\kern-.025em b}\kern-.08em
    T\kern-.1667em\lower.7ex\hbox{E}\kern-.125emX}}
\begin{document}

\title{Quantum-resistant digital signatures schemes for low-power IoT}

\author{\IEEEauthorblockN{1\textsuperscript{st} Hannes Hattenbach}
\IEEEauthorblockA{\textit{Computational Science} \\
\textit{Freie Universität}\\
Berlin, DE \\
hannes.hattenbach@fu-berlin.de}
}

\maketitle

\begin{abstract}
%*CRITICAL: Do Not Use Symbols, Special Characters, Footnotes, or Math in Paper Title or Abstract.
\end{abstract}

\begin{IEEEkeywords}
Internet of Things, Quantum Resistance, Secure Signatures, Power Constraint Devices
\end{IEEEkeywords}

%TODO 12 Seiten

\section{Introduction}
The quantum revolution is coming. With quantum computers\footnote{compare section \ref{quantum_computing}} on the way to get more and more functional, people are fearing a loss of their security and privacy.
That is because there are algorithms based on Shors algorithm that can forge signatures and decrypt encrypted messages whos security is based on discrete logarithms, including elliptic curves or prime factoriztion, like our most common schemes ECDSA and RSA respectively are.
The quantum computer only needs access to the public keys of these asymetric schemes.
The expenditure to forge a signature\footnote{that is considered secure under normal circumstances 
%TODO: more precise? Unforgeablu under chosen blabla
} with classic\footnote{we refer to classic if something is not directly leveraging entanglement or superposition} computers rises exponentially with increased key length, therefor beeing essentially unbreakable by classic computers.
A sufficient quantum computer on the other hand can derive a private key from a public key in polynomial time, therefor rendering these schemes broken.

That is why there are currently schemes under standarization\cite{PQClean-GH} that are based on other hard problems (not number theory) like so called lattice problems that cannot be that easily forged by quantum computers to save our privacy and security.

One of the use cases not directly comming to mind for the end user, but beeing as important non the less is signing sensitive sensor data in the Internet of Things (IoT).
Another problem coming up in the IoT compared to end-user-devices like Laptops and Smartphones though is the severe ressource contraintness. 
The IoT consist of low power devices with wery few storage and computing power.

In this paper i am going to evaluate existing signature schemes and their usage possibilities for the IoT regarding their performance metrics.

Therefor i am going to give a small introduction and background to quantum computing, beeing a little more detailed about their ability to break current encryption and signature standards.
In the next section i will give an overview over current candidates for Quantum Resistant (QR) Algorithms and giving performance metrics for those.
The following chapter will then focus on signature schemes in the IoT, starting with additional performance metrics relevant in the IoT.
With a little more details about two failed signature schmemes to highlight potetial pitfalls. 
And finally focussing on the best signature contenter for the IoT so far: FALCON.


% TODO ? Background section to put this in?
\section{Internet of Things}
%discussion what the boundries of iot compare different devices from linux devices rPI over 32bit Arm based processors with RTOS to 8bit microcontrollers like arduino/atmega
The IoT consists of devices of all sorts, having in common, that they communicate with each other and the enviroment rather than directly with humans.
Those devices range from automatic lights and smart home devices to tiny interconnected sensors in automatic fabrication.
A common characteristic though is, that most of these devices have limited processing power, flash storage and random access memory (RAM). 
A popular example for hobbiest IoT devices is the ESP32 from Espressif Microsystems.
They offer multiple Modules with up to 240Mhz Clock on the 32 IC, up to 16MB Flash Storage and 320KiB RAM.
Which is more than other compareable devices but way less then a lower spec modern smartphone, with 10 times the frequency, 4GB of RAM and 64GB of storage.


\section{Quantum Resistant Security}
\subsection{Quantum Computing}\label{quantum_computing}
In contrast to classical computers, where information is processed in discete states, a quantum computer leverages quantum mechanics to operate on so-called qubits - quantum objects that can be in superposition or entangled with each other. 
Opening a new kind of computing. 
One of implications of that is, that it is now possible to factor large numbers in polynomial time \cite{Shor} \footnote{Shors algorithm uses a so-called Quantum-Fourier-Transform (QFT) to (probablisticly) get the frequencys of which a given function output occurs. That can be used together with euclids algorithm of finding the greatest common devisor to derive the prime factors}. 
Prior to to quantum computers this was considered a hard problem that could only be computed in exponential time and was therefor considered practically impossible and was used as the basis-problem for RSA encryption.
Similar to that other common schemes like ECDSA can also be broken be slightly modidied versions of Shors Algorithm.
\subsection{QR Algorithms}
The two main algorithms with practical use cases that have a great speed-up compared to classical solutions, are the already introduced algorithm by Shor and an algorithm by Grover taht can essentially reverse one-way functions.
While Shors algorithm provides exponential speed-up Grovers algorithm only provides quadratic speed up. It was also shown, that something similar to grovers algorithm but with exponential speedup is impossible \cite{Strengths&Weaknesses_QC}. Which implies that Hashing as well as symetric cryptography stays relatively secure.
The quadratic speedup provided by quantum computers can easily be mitigated by doubling the key length.
On the other hand though, classical asymetric cryptography is endangered by shors algorithm and quantum computers.

But not all asymetric cryptography schemes are equally affected.
There are different proposals, both for QR encryption and for QR signature schemes.
They all do have in common though, that their security is not absolutely mathematically prooven, but based upon assumptions.
We therefor need to consider a few measures that make schmemes more or less secure.
\subsubsection{Performance Metrics}
Some performance metrics exist in QR schemes as well as in classic schemes.
Keylength and
key exchange message length \cite{QR_algs} are the more obvious ones.
The computing time also comes to mind as a perfomance metric. Here you need to differentiate between key generation, which is less important, since it should only occur rarely, and signing as well as signature verification \footnote{as well as its counterparts de- and encryption}.

Primarely in signatures another metric arises: how often can a private key be used before it needs to be switched out for another one, because the signature leaked information of the key.
This is not particularly relevant in most cases, as methods can be used to create long term procedures from short term procedures (those where a key can rarely, if ever, be recycled).
But it is relevant in the case of the IoT, since those methods require extra memory which is sparse in IoT-devices. Adiitionally they tend to make the signatures themself longer, which also is not preferrable in the IoT. \cite{QR_algs}

\subsubsection{Encryption}
knapsack problem - broken
''
conjugacy search problem and related problems in braid groups, and the problem of solving
multivariate systems of polynomials in finite fields
'' also mostly broken or badly understood \cite{QR_algs}

code based:
- McEliece Error correction codes transformed - secure and fast (100micros) but keys are k*n matrices : millions of bits \cite{QR_algs} - not feasable 

\subsubsection{Signatures} 
hash based or lattice based:
- lamport OT signature \cite{QR_algs} (as the name implies only useable one time - useless ; can precompute a bunch that are verifiable with the same pubk - useless since rare storage)
- lattice based signatures (high-d basis find shortest vector (SVP) or closest lattice vector to arbitrary point (CVP)) require megabits of basis - unfeasable NTRU reduce to kilobits by introducing symetry \cite{QR_algs}
    -10-100* faster than conventional crypto
    Encryption : - vulnerable to CCA , lattice reduction techniques - padding scheme [30] , longer keys \cite{QR_algs}
    Signatures : - map message to vector, sign by solving (CVP) - leaks information about PrivK - broken after 400messages - dont give closest vektor but a close enough one - secure for 1 mrd (billion) sigs (still adviced to swap after 10mil)

- McEliece decrypt padded message digest - try thousands of paddings - signing takes 30seks, 4mb priv/pub KEY-size -not feasbale
\section{QR Signatures in IoT}
\subsubsection{Performance Metrics in IoT}
- key/ exchange message/ signature size 
    - cache/ ram usage
    
- setup(ms)  lifetime, pubk size, privk size, sig

\subsection{Failed Signatures}
\subsubsection{WalnutDSA}
\subsubsection{qTESLA}
\subsection{FALCON}

\section{Conclusion}

\bibliographystyle{IEEEtran}
\bibliography{lit.bib, lit_specific.bib}

\end{document}
